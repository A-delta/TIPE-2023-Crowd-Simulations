\section{Construction du modèle}

\begin{frame}
    \frametitle{Modélisation}
    \framesubtitle{Plan}
    \textbf{Modélisation}
    \begin{itemize}
        \item <2-> Modèle mathématique simple
        \item <3-> Adaptation informatique
        \item <4-> Amélioration du modèle
    \end{itemize}
\end{frame}


\begin{frame}
    \frametitle{Modèle mathématique}
    \framesubtitle{Simplifications}
    On restreint l’étude à deux dimensions, en \textbf{discrétisant} le temps et le plan :
    \bigskip
    \begin{itemize}
        \item <2-> Carte $\leftrightarrow P_{i,j} \in M_{n,p}(\mathbb{N}), (n, p) \in \mathbb{N}^2$
        \item <3->  Agent $ \leftrightarrow position, but \in \llbracket 1, n\rrbracket \times \llbracket 1, p\rrbracket $
    \end{itemize}

\end{frame}



\begin{frame}[fragile]
    \frametitle{Modèle mathématique}
    \framesubtitle{Implémentation des structures}
    \begin{code}
        \begin{minted}[linenos]{c}
struct location {
  int y;
  int x;
}

struct person {
  struct location pos;
  struct location goal;
}
\end{minted}
    \end{code}
\end{frame}


\begin{frame}[fragile]
    \frametitle{Modèle mathématique}
    \framesubtitle{Implémentation des structures}
    \begin{code}
        \begin{minted}[linenos]{c}
struct map {
int **level;      // plan de la simulation
int start_nb;     // nombre d'entrées
location *starts; // et leurs positions
int exit_nb;      // nombre de sorties
location *exits;  // et leurs positions
int width;        // largeur
int height;       // hauteur
}
\end{minted}
    \end{code}
    \[ map \rightarrow level[i][j] \leftrightarrow \text{ nombre de personnes à la position } (i,j)\]
\end{frame}


\begin{frame}
    \frametitle{Modèle mathématique}
    \framesubtitle{Squelette du programme}
    On définit des fonctions :\\
    \begin{itemize}
        \item Simulation
              \begin{itemize}
                  \item <2-> $charger\_carte$
                  \item <3-> $intention$
                  \item <4-> $deplacement$
              \end{itemize}
              \bigskip
        \item <5-> Génération des résultats
              \begin{itemize}
                  \item <6-> $creer\_image$
                  \item <7-> $sauvegarder\_image$
              \end{itemize}
    \end{itemize}
\end{frame}


\begin{frame}
    \frametitle{Modèle mathématique}
    \framesubtitle{Précisions}

    Notant \(\mathbb{A}\) l'ensemble des agents de la simulation, $intention$ et $deplacement$ se définissent par :

    \begin{empheq}[left=intention \colon \empheqlbrace]{align*}
        \mathbb{A} \to& \mathbb{N}^2\\
        x \mapsto& intention(x)
    \end{empheq}

    \begin{empheq}[left=deplacement \colon \empheqlbrace]{align*}
        M_{n,p}(\mathbb{N}) \times \mathbb{A} \to& M_{n,p}(\mathbb{N})\\
        (M, x) \mapsto& deplacement(M, x)
    \end{empheq}
\end{frame}

